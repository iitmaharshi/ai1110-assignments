%%%%%%%%%%%%%%%%%%%%%%%%%%%%%%%%%%%%%%%%%%%%%%%%%%%%%%%%%%%%%%%
%
% Welcome to Overleaf --- just edit your LaTeX on the left,
% and we'll compile it for you on the right. If you open the
% 'Share' menu, you can invite other users to edit at the same
% time. See www.overleaf.com/learn for more info. Enjoy!
%
%%%%%%%%%%%%%%%%%%%%%%%%%%%%%%%%%%%%%%%%%%%%%%%%%%%%%%%%%%%%%%%


% Inbuilt themes in beamer
\documentclass{beamer}

% Theme choice:
\usetheme{CambridgeUS}

% Title page details: 
\title{Probability class 12} 
\author{Maharshi Kadeval}
\date{\today}
\logo{\large \LaTeX{}}

\providecommand{\brak}[1]{\ensuremath{\left(#1\right)}}
\begin{document}

% Title page frame
\begin{frame}
    \titlepage 
\end{frame}

% Remove logo from the next slides
\logo{}


% Outline frame
\begin{frame}{Outline}
    \tableofcontents
\end{frame}


% problem frame
\section{Problem statement}
\begin{frame}{Problem statement}
\begin{enumerate}
\item if $y = \sqrt{X}$ and X is an exponential random variable, show that Y represents a Rayleigh random variable\\
\end{enumerate}

\end{frame}
\section{Solution}
\begin{frame}{Solution}

X has a probability density function:
\begin{align}
f_X\brak{x} &= \frac{1}{\alpha} e^{-\frac{x}{\alpha}}
\end{align}
The transformation $Y = g\brak{X} = \sqrt{X}$ is a 1-1 transformation from $X = \lbrace x|x>0 \rbrace$ to $Y = \lbrace y|y>0 \rbrace$ with inverse $X = g^{-1}\brak{Y} = Y^{2}$ \\
and jacobian $\frac{dX}{dY} = 2Y$\\
\end{frame}
\begin{frame}
Therefore by the transformation technique, the probability density function of Y is:
\begin{align}
f_Y\brak{y} &= f_X\brak{g^{-1}\brak{y}} \bigg | \frac{dX}{dY} \bigg | \\
&= \frac{1}{\alpha} e^{-\frac{y^{2}}{\alpha}} |2y|\\
&= \frac{2y}{\alpha} e^{-\frac{y^{2}}{\alpha}} 
\brak{y>0} 
\end{align}

which is the probability density function of a Rayleigh random variable\\
\begin{center}
Hence Proved
\end{center}

\end{frame}
\end{document}