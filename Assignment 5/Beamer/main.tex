%%%%%%%%%%%%%%%%%%%%%%%%%%%%%%%%%%%%%%%%%%%%%%%%%%%%%%%%%%%%%%%
%
% Welcome to Overleaf --- just edit your LaTeX on the left,
% and we'll compile it for you on the right. If you open the
% 'Share' menu, you can invite other users to edit at the same
% time. See www.overleaf.com/learn for more info. Enjoy!
%
%%%%%%%%%%%%%%%%%%%%%%%%%%%%%%%%%%%%%%%%%%%%%%%%%%%%%%%%%%%%%%%


% Inbuilt themes in beamer
\documentclass{beamer}

% Theme choice:
\usetheme{CambridgeUS}

% Title page details: 
\title{Probability class 12} 
\author{Maharshi Kadeval}
\date{\today}
\logo{\large \LaTeX{}}

\providecommand{\brak}[1]{\ensuremath{\left(#1\right)}}
\begin{document}

% Title page frame
\begin{frame}
    \titlepage 
\end{frame}

% Remove logo from the next slides
\logo{}


% Outline frame
\begin{frame}{Outline}
    \tableofcontents
\end{frame}


% problem frame
\section{Problem statement}
\begin{frame}{Problem statement}
\begin{enumerate}
\item Given three identical boxes I, II and III, each containing two coins. In
box I, both coins are gold coins, in box II, both are silver coins and in the box III, there
is one gold and one silver coin. A person chooses a box at random and takes out a coin.
If the coin is of gold, what is the probability that the other coin in the box is also of gold?\\
\end{enumerate}

\end{frame}


% solution frame
\section{Solution}
\begin{frame}{Solution}
    Let $X_{1} \in \lbrace 1,2,3 \rbrace$ and $X_{2} \in \lbrace 0,1 \rbrace$ be random variables which have the following meanings associated with them :\\
\begin{block}{description of events}
$X_{1} = 1$ : Bag 1 is selected\\
$X_{1} = 2$ : Bag 2 is selected\\
$X_{1} = 3$ : Bag 3 is selected\\
$X_{2} = 0$ : gold coin is selected\\
$X_{2} = 1$ : silver coin is selected\\

\end{block}
Therefore,the required probability is $P\brak{X_{1}=1|X_{2}=0}$.\\

\end{frame}
 \begin{frame}
 By Baye's Theorem and total probability,\\


\begin{align}
P\brak{X_{1}=1|X_{2}=0}&= \frac{P\brak{X_{1}=1} P\brak{X_{2}=0|X_{1}=1}}{\Sigma^{3}_{i=1} P\brak{X_{1}=i}P\brak{X_{2}=0|X_{1}=i}}\\
&= \frac{\frac{1}{3} \times 1}{\frac{1}{3} \times 1 + \frac{1}{3} \times 0 + \frac{1}{3} \times \frac{1}{2} }\\
&= \frac{2}{3}\\
\therefore P\brak{X_{1}=1|X_{2}=0} &= \frac{2}{3}
\end{align}

 \end{frame}
 


\end{document}